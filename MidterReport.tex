\documentclass[a4paper,10pt]{article}
\usepackage[utf8]{inputenc}
\usepackage[T1]{fontenc}
\usepackage{authblk}
\usepackage{physics}
\usepackage{graphicx}
\usepackage{url}

%\newcommand{\todo}[1]{\color{red}\textbf{\hl{#1}}\color{black}\xspace}
%\newcommand{\todo}[1]{}

\title{Midterm Report: Quantum Fourier Transform}
\author[]{Md Maruf Hossain \\ mhossa10@uncc.edu}
\affil[]{Department of Computer Science, UNC Charlotte}
\date{}

\begin{document}
\maketitle
\textit{Fourier Transform} decomposes or analysis a function of time or signal into its constituent frequencies. 
\textit{Quantum Fourier Transform}{QFT} is a critical part of \textit{Shor}'s Algorithm and many other algorithms as well.
The main idea of this report is to explain the \textit{Quantum Fourier Transform}, that can help to understand the QFT. This document is a midterm report of 
Quantum Computing class, that is why most the information comes from class lecture, documents and from web sites~\cite{quantumbook}.

\section{Quantum Fourier Transform(QFT)}
Fourier analysis is a system or method that can help to describe the internal frequencies of a function. \textit{Quantum Fourier Transform} 
is a quantum implementation of the discrete Fourier transform. Let say, you have a $n$ qubit state vector $\ket{\alpha}\ =\ \alpha_0\ket{0} + 
\alpha_1\ket{1} + \cdots +\alpha_n\ket{n}$, now QFT algorithm will transform it into $\ket{\beta}\ =\ \beta_0\ket{0} + \beta_1\ket{1} + \cdots + \beta_n\ket{n}$ 
that a measurement on $\ket{\beta}$ will only return one of its $n$ components. 

The main difference between \textit{Hadamard} transform and QFT is that it introduces phase, that is called primitive roots of unity $\omega$. It is important to 
understand the concept of primitive roots of unity $\omega$ before going for the \textit{Quantum Fourier Transform}. We know that $x^n = 1$ has exactly $n$ solutions. 
Let say $n=2$ then $x$ could be 1 or -1. Now, if $n=4$ then $x$ could be $1,i,-1\ or\ -i$. Here $i$ is a complex number and these are the roots of the equation. 
The interesting fact is that you can write these roots by the powers of $\omega\ =\ e^{2\pi i/n}$ and $\omega$ is called a primitive $n^{th}$ root of unity. 
  
\begin{figure}[htp!]
  \centering
  \includegraphics[width=.4\linewidth]{unity_of_roots.png}
  \caption{The 5 complex $5^{th}$ roots of 1.~\cite{quantumbook}}
  \label{fig:unity_of_roots}
\end{figure}

We can explain $\omega$ by the unit circle, figure ~\ref{fig:unity_of_roots} represents the roots of unity for $n=5$. It shows the different phase of $\omega$ along with 
the other complex roots. We can see $\omega$ lies on the unit circle that means $|\omega|$ represent the absolute value of the radius, $|\omega|\ =\ 1$. On the other hand, 
the line from the center to the $\omega$ makes an angle with the horizontal original line, $\phi\ =\ 2\pi/M$. Here, M is the M\textit{th} root of unity. Now the relation between 
$\omega$ and $\phi$ is that if you square the $\omega$ then the angle $\phi$ will be double. In general, if you apply j\textit{th} power on the $\omega$ that means $\omega^{j}$ 
then the phase angle turn into $\phi\ =\ 2j\pi/M$. We can represent M as $2^m$ then we can get $\phi\ =\ 2j\pi/2^m$.

So far we get some knowledge about the roots of unity and phase angle, now we can move forward towards the \textit{Quantum Fourier Transform}. A matrix is the best way to 
represent a \textit{Quantum Fourier Transform} system, that is why to represent a $2^m$ dimensional QFT you need $2^m\ \times\ 2^m$ matrix $F_{2^m}$. QFT matrix function 
can be defined by the 
\begin{eqnarray*}
F_{2^m}[i,j]\ =\ \omega^{ij}/\sqrt{2^m}
\end{eqnarray*}    
where $\omega\ =\ e^{2\pi i/2^m}$ is the $2^m$-\textit{th} root of unity. So we will get a matrix like,
\begin{eqnarray*}
QFT_{2^m}=\frac{1}{\sqrt{2^m}}
\begin{pmatrix}
1 & 1 & 1 & \cdots & 1 \\
1 & \omega & \omega^2 & \cdots & \omega^{2^m-1} \\
1 & \omega^2 & \omega^4 & \cdots & \omega^{2(2^m-1)} \\
\vdots & \vdots & \vdots & \ddots & \vdots \\
1 & \omega^{2^m-1} & \omega^{2(2^m-1)} &\cdots &\omega^{(2^m-1)^2}
\end{pmatrix}
\end{eqnarray*}

The main idea of this transformation is to take the vector 
$\begin{pmatrix}
  \alpha_0 \\
  \alpha_1 \\
  \vdots \\
  \alpha_{n-1} 
\end{pmatrix}
$ and transform them into 
$\begin{pmatrix}
  \beta_0 \\
  \beta_1 \\
  \vdots \\
  \beta_{n-1} 
\end{pmatrix}
$
Now, if we choose $M = 2$ that means,  $2^m = 2$, $\omega\ =\ 2^{\pi i}$ and phase angle $\phi\ =\ \pi$. So, from the unity circle $\omega^0\ =\ 1$ and $\omega^1=-1$ 
because of the angle shift $\pi$.
\begin{eqnarray*}
QFT_2\ =\ \frac{1}{\sqrt{2}}
\begin{pmatrix}
 1 & 1 \\
 1 & \omega
\end{pmatrix}
\ =\ 
\begin{pmatrix}
  1 & 1 \\
  1 & -1
\end{pmatrix}
\end{eqnarray*}

Similarly, if we choose $M=4$ that means $2^m=4$ then phase angle will be $\phi\ =\ \pi/2$. So, from the unity of circle we can get 
$\omega^0=1,\ \omega^1=\omega=i,\ \omega^2=-1,\ \omega^3=-i,\ \omega^4=1,\ \omega^5=i,\ \omega^6=-1,\ \omega^7=-i,\ \omega^8=1,\ \omega^9=i$

\begin{eqnarray*}
F_4\ =\ _QFT_{2^2}\ =\ QFT_{4}\ =\ \frac{1}{\sqrt{2}}
\begin{pmatrix}
 1 & 1 & 1 & 1 \\
 1 & \omega & \omega^2 & \omega^3 \\
 1 & \omega^2 & \omega^4 & \omega^6 \\
 1 & \omega^3 & \omega^6 & \omega^9
\end{pmatrix}
\ =\ 
\begin{pmatrix}
  1 & 1 & 1 & 1 \\
  1 & i & -1 & -i \\
  1 & -1 & 1 & -1 \\
  1 & -i & -1 & i 
\end{pmatrix}
\end{eqnarray*}

Now, if we move column(0 and 2) to the left then we will get,
\begin{eqnarray*}
F'_4\ =\
\begin{pmatrix}
  1 & 1 & 1 & 1 \\
  1 & -1 & i & -i \\
  1 & 1 & -1 & -1 \\
  1 & -1 & -i & i 
\end{pmatrix}
\ =\ 
\begin{pmatrix}
 H & AH \\
 H & -AH
\end{pmatrix}
\end{eqnarray*}

Where $A\ =\ 
\begin{pmatrix}
 1 & 0 \\
 0 & i
\end{pmatrix}
$ is a phase shift operation. At the beginning of the article, we mentioned that the main difference between QFT and \textit{Hadamard} is the phase. 
In general, we can write,
\begin{eqnarray*}
F_{2^m}\ =\ \frac{1}{\sqrt{2}}
\begin{pmatrix}
 F_{2^{m-1}} & AF_{2^{m-1}} \\
 F_{2^{m-1}} & -AF_{2^{m-1}}
\end{pmatrix}
\end{eqnarray*}
where
\begin{eqnarray*}
A\ =\ 
\begin{pmatrix}
 1 & & & \\
   & \omega & & \\
   & & \ddots & \\
   & & & \omega^{2^{m-1}-1}
\end{pmatrix}
\end{eqnarray*}
Now, m qubits require $2^m$ operations to compute $F\ket{\emptyset}$and a \textit{Fast Fourier Transform} can compute $A\ket{X}$ in $O(m2^m)$ steps.
But, with the \textit{Quantum} circuit speed up, we can implement $F_{2^m}\ket{\emptyset}$ with a \textit{quantum circuit} of $O(m^2)$ gates. As earlier we mentioned, 
this calculation has a great significance on the \textit{Shor}'s algorithm that is why this enhancement is also important.
\bibliographystyle{unsrt}
\bibliography{midref}


\end{document}
